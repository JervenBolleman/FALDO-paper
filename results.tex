\section*{Results}
One of the practical goal driving the development of FALDO was to be able
to represent all the annotated sequences in UniProt and the INSDC as RDF
triples, as a step towards providing this data via SPARQL end points whereby
it can be queried.
UniProt is already available as RDF, and a revised version of the full
dataset using FALDO has already been tested and is expected to be included
in the UniProt 2013\_11 release \textit{TODO - fill in version and data}.

The protein examples considered, such as the UniProt feature annotations,
require relatively simple locations within protein sequences to be described (see Fig:\ref{fig:UniProt}).
There are corner cases for situations like unknown boundaries for these we use the $faldo:InRangePosition$.

\begin{figure}
\begin{verbatim}
AC   P05064;
...
FT   ACT_SITE    230    230       Schiff-base intermediate with
FT                                dihydroxyacetone-P.
\end{verbatim}
\begin{verbatim}
uniprot:P05064 a up:Protein ;
   up:annotation SHA:E50..1B .
SHA:E50..1B a up:Active_Site_Annotation ;
   rdfs:comment ``Schiff-base intermediate with dihydroxyacetone-P''
   faldo:location range:22571003137111085tt230tt230 .
range:22571003137111085tt230tt230 a faldo:Position ;
   faldo:position 230 ;
   faldo:reference isoform:P05064-1 .
\end{verbatim}
\caption{Example from UniProt entry P05064 showing the position of an active site in both faldo and the UniProt flat-file format.}
\label{fig:UniProt}
\end{figure}
\subsection*{Complement strand}
Feature locations on nucleotide sequences can be relatively simple,
%GenBank
for the example (shown in Fig:\ref{fig:insdcComplement}) the \textit{cheY} gene in
\textit{Escherichia coli} str. K-12 substr. MG1655 (accession NC\_000913.2)
is described in the INSDC feature table as \texttt{complement(1965072..1965461)},
which is 390 base pairs using inclusive one-based counting.
This feature begins on the base complementary to $start = 1965461$
and finishes at $end = 1965072$, and so the INSDC location string
can be interpreted as \texttt{complement($end$..$start$)},
FALDO respects this biological interpretation of a feature location
on the reverse strand:

%Notice begin, end = 1965461, 1965072 (biological interpretation)
\begin{figure}
\begin{shaded}
\small
\begin{verbatim}
# 
@prefix faldo: <http://biohackathon.org/resource/faldo#> .

# Here are example triples describing a gene, note the location line:
<_:geneCheY> a so:Gene ;
           rdfs:label "cheY" ;
           faldo:location <_:example> ;

# The following triples define the location used by the feature above,
# made up of a region and its two positions:
<_:example> a faldo:Region ;
           faldo:begin <_:example_b> ;
           faldo:end <_:example_e> .

<_:example_b> a faldo:Position ; 
           a faldo:ExactlyKnownPosition ;
           a faldo:ReverseStrandPosition ;
           faldo:position "1965461"^^xsd:int ;
           faldo:reference refseq:NC_000913.2 .

<_:example_e> a faldo:Position ; 
           a faldo:ExactlyKnownPosition ;
           a faldo:ReverseStrandPosition ;
           faldo:position "1965072"^^xsd:int ;
           faldo:reference refseq:NC_000913.2 .
\end{verbatim}
\end{shaded}
\caption{Incomplete example of using Faldo in turtle format to describe
gene chrY being at complement(NC\_$000913.2:1965072..1965461$)}
\label{fig:insdcComplement}
\end{figure}

In contrast, other formats like GFF require $start \leq end$
regardless of the strand, which is equivalent to interpreting
the INSDC location string as \texttt{complement($start$..$end$)}.
This convention has a number of practical advantages when
dealing with numerical operations on features sets, such as
checking for overlaps or indexing data. For example, the
feature length is given by $length = end - start + 1$ under
this numerically convenient scheme where the interpretation
of $start$ versus $end$ is strand independent.



\subsection*{Compound locations}
Nucleotide data from the INSDC (or GFF3) can require considerably
more complicated sequence locations built out of multiple-regions.

\textit{TODO - examples}.

\subsection*{Fuzzy locations}
Nucleotide data from the INSDC can also require ambiguous end points.

\textit{TODO - examples}.

All of the UniProt protein sequences are now available in RDF format,
with a SPARQL end point currently in beta. Converting more recent
curated INSDC data has been straightforward, however there are a
number of deprecated location strings which have been problematic.
\textit{TODO - examples? manual curation to fix last few oddities?}

\subsection*{Restriction enzymes}

The recognition sites of most restriction enzymes are relatively
straightforward to describe, as is the cleavage site of a blunt
end cutting enzyme \textit{TODO - example of this? Between location}.

However, the cutting point of sticky-end cutting enzymes which
leave an overhang are more challenging to describe. Here the
flexibility in FALDO to allow the start and end positions of a region
to have different strands can be used (Figure~\ref{fig:HindIII}).

\begin{figure}
\begin{framed}
\small
\begin{verbatim}
    \/
5'..AAGCTT..3' 
    123456
3'..TTCGAA..5'
        /\
        
:EnzymeRestrictionSite a example:EnzymeRestrictionSite ;
       faldo:location :EnzymeRestrictionRegion .
       
:EnzymeRestrictionRegion a faldo:Region ;
       faldo:begin :53_1 ;
       faldo:end   :35_6 .
       
:53cleaveageSite a example:CleavageSite ;
       faldo:location :53cleaveageSitePosition .
:35cleaveageSite a example:CleavageSite ;
       faldo:location :35cleaveageSitePosition .
        
:53cleaveageSitePosition a faldo:InBetweenPosition, SO:0001689  ; #five_prime_restriction_enzyme_junction 
       faldo:after :53_1 ;        
       faldo:before :53_2 .
       
:35cleaveageSitePosition a faldo:InBetweenPosition, SO:0001690 ; #three_prime_restriction_enzyme_junction 
       faldo:before :35_5 ;
       faldo:after :35_6 . #Note the before after reversion due to biological start      

       
:53_1 a faldo:ExactPosition , faldo:ForwardStrandedPosition ;
      faldo:position 1 .

:53_2 a faldo:ExactPosition , faldo:ForwardStrandedPosition ;
      faldo:position 2 .

:35_5 a faldo:ExactPosition , faldo:ReverseStrandedPosition ;
      faldo:position 5 .

:35_6 a faldo:ExactPosition , faldo:ReverseStrandedPosition ;
      faldo:position 6 .
\end{verbatim}
\end{framed}
\caption{HindIII restriction enzyme cleavage site with sticky ends}
\label{fig:HindIII}
\end{figure}

Circular genomes bring have number of interesting features to describe. 
The most common is  a feature that straddles the origin of replication or the '0' mark.
For example a gene encoded on the reverse strand of the f1 bacteriophage \textit{todo find accession} which is 6407 base pairs long.
GeneII transcription starts from coordinate 6006 and the transcription ends at position 831.


\subsection*{More examples}

\textit{TODO - other ideas:}
\begin{itemize}
\item Example of Cys bonding between one or two proteins?
\item RNA/DNA digest sites, and the cut sites -- especially for over hangs (start and ends on opposite strands)
\item tRNA hairpin (two complementary regions)?
\item $\beta$-sheet in protein?
\item RNA-Seq mapping (useful for SAM/BAM)?
\end{itemize}



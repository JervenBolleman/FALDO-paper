%\documentclass{llncs}
%\NeedsTeXFormat{LaTeX2e}[1995/12/01]
\documentclass[10pt]{bmc_article}
\usepackage{cite} % Make references as [1-4], not [1,2,3,4]
\usepackage{url}  % Formatting web addresses  
\usepackage{ifthen}  % Conditional 
\usepackage{multicol}   %Columns
\usepackage[utf8]{inputenc} %unicode support
\usepackage{listings}
%\usepackage[applemac]{inputenc} %applemac support if unicode package fails
%\usepackage[latin1]{inputenc} %UNIX support if unicode package fails
\urlstyle{rm}


\def\includegraphic{}
\def\includegraphics{}



\setlength{\topmargin}{0.0cm}
\setlength{\textheight}{21.5cm}
\setlength{\oddsidemargin}{0cm} 
\setlength{\textwidth}{16.5cm}
\setlength{\columnsep}{0.6cm}

\newboolean{publ}
\newenvironment{bmcformat}{\begin{raggedright}\baselineskip20pt\sloppy\setboolean{publ}{false}}{\end{raggedright}\baselineskip20pt\sloppy}

\lstdefinelanguage[1.1]{sparql}{morekeywords={PREFIX,SELECT,CONSTRUCT,WHERE,ASK,DESCRIBE,A,NOT,EXISTS,FILTER,regex},sensitive=true,morestring=[s]{<>},morestring=[s]{""},escapeinside={~}{~},extendedchars=true,morecomment=[l]{\#}}
\lstset{
  basicstyle=\small, % print whole listing small
  identifierstyle=, % nothing happens
  commentstyle=\color{blue}, % white comments
  stringstyle=\ttfamily, % typewriter type for strings
  showstringspaces=false
  numbers=left, 
  numberstyle=\tiny,
  numberbychapter=true,
  xleftmargin=0.1\linewidth,
  xrightmargin=0.1\linewidth,
  boxpos=c,
  frame=, 
  numbersep=2p,
  columns=fullflexible,
  escapechar=`
}

%\newfloat{example}{h}{ttl}[section]
%\floatname{example}{Example}		
\begin{document}
\begin{bmcformat}
\title{Semantic standard for describing the location of nucleotide and protein feature annotation.}
\author{Jerven Bolleman\correspondingauthor$^1$ \email{Jerven Bolleman\correspondingauthor - jerven.bolleman@isb-sib.ch} 
         \and Robert Beuels$^2$%
         \and Robert Hoehndorf$^3$%
         \and Peter Cock$^3$ %
		 \and Toshiaki Katayama$^4$ %
		 \and Raoul JP Bonnal$5$ %
		 \and Michel Dumontier$^5$ %
		 \and Takatomo Fujisawa$^6$ %
		 \and Francesco Strozzi$^7$ %
		 \and Joachim Baran$^8$ %
         }
\address{
    \iid(1)SIB Swiss Institute of Bioinformatics, Centre Medical Universitaire, 1 rue Michel
Servet, 1211 Geneva 4, Switzerland,
 \iid(2) fill in,
 \iid(3) James Hutton Institute, Invergowrie, Dundee DD2 5DA, UK,
 \iid(4) fill in,
 \iid(5) Integrative Biology Program, Istituto Nazionale Genetica Molecolare, Milan, Italy,
 \iid(6) fill in,
 \iid(7) fill in, and
 \iid(8) fill in.
        }
\maketitle

\begin{abstract}
\begin{description}
\item[Background] Nucleotide and protein sequence feature annotation is essential to understand our biology on the genome, transcriptome, and proteome level. Unfortunately, there is no standard that describes this potentially complex location information as
subject-predicate-object triples suitable for semantic web queries using RDF/SPARQL.
\item[Description] We developed a schema ontology to describe the positions of annotated features that is usable for describing glycan binding sites, protein annotation and nucleotide features in sequence records.
Using the same data format to represent sequence positions independent of files allows us to integrate sequence data from multiple sources and data types.
The integration capabilities are shown by JBrowse using multiple SPARQL endpoints to display genomic feature annotation as well as protein annotation from UniProt mapped to genomic locations.
\item[Conclusions] This standard ontology allows users to merge sequence annotation from multiple sources using federalised SPARQL queries against public endpoints and local private datasources. At the same time the ontology is expressive enough to describe all known biological use cases accurately.
\end{description}
\end{abstract}

\section*{Keywords}

\section*{Background}
Describing regions of biological sequences is a vital part of genome and protein sequence
annotation, and in areas beyond this such as describing modifications such methylation
of nucleotide sequences, or glycosylation of proteins.
There are multiple different conventions for storing this kind of information in
plain text flat file formats such as GTF, GFF3, GenBank and EMBL,
and more structured domain specific XML formats such as the INSDC or UniProt XML.
As a result, file format interconversion is a common background task in current script-centric
bioinformatics pipelines, a necessary evil for combining tools supporting different formats or
format variables.

In the growing area of semantic web technologies, it was recognised at the BioHackathon 2012
meeting \cite{BioHack2011and2012} that a failure to standardise how to describe positions
and regions on biological sequences would be an obstacle to the adoption of federalized
SPARQL/RDF queries which have the potential to enable cross-database queries and
analyses. Discussion and prototyping with representatives from major sequence databases
such as UniProt, DDBJ (part of the INSDC partnership with the NCBI and EMBL-Bank),
and major glycomics databases \textit{(TODO - which ones? introduce later? e.g.
 Bacterial Carbohydrate Structure Database (BCSDB), GlycomeDB,
 GLYCOSCIENCES.de,
 Japan Consortium for Glycobiology and Glycotechnology Database (JCGGDB),
 MonosaccharideDB,
 Resource for INformatics of Glycomes at Soka (RINGS),
 and UniCarbKB)}
\textit{(TODO - and PDBj too?)} and assorted open source developers during this meeting
led to the development of the Feature Annotation Location Description Ontology (FALDO).

Understanding how nucleotide and protein sequences in databases correspond to the real world is key to understanding biological processes.
Determined nucleotide sequences are stored in a large number of databases,
such as GenBank/DDBJ/EMBL and vast numbers of new sequence records are generated every day since the advent of high throughput sequencing.
Currently there are a vast number of ways of describing such sequences and their biological relation e.g. GenBank format, GFF3 and variants.
All of these formats require the annotation to be stored in the same file as the sequence,
making it difficult to merge information between files without copying all data.

While determining sequences has become vastly cheaper over time basic curation of annotation has not.
This means that we still move most annotation from one sequence to another,
e.g. gene annotation moves from one genome assembly to the next.

The proposed standard allows us to accurately describe the position of a feature on multiple sequences.
For example a gene can start at position $X$ on genome assembly $A$,
while conceptually the same gene is positioned on $X'$ on genome assembly $A+1$.
The proposed standard also deals with incomplete assemblies as well as protein sequence annotation such as found in UniProtKB.

None of the above tools or formats are flexible enough to discuss all of genetics or proteomics. 


\section*{Implementation}
Some tech stuff here, for example we use one-based counting
with respect to the reference sequence's forward strand
(where the reference is a nucleotide sequence).

Probably a good place to talk about strand representations?
See Figure~\ref{fig:strands}.

\begin{figure}[p]
\begin{center}
\includegraphics[width=10cm]{figures/figure-strand.pdf}
\end{center}
\caption{Assorted conventions for regions, start, end, and strands.
This figure shows two hypothetical features on a DNA sequence
(labelled \texttt{chr1}), on either the forward strand (orange) or
reverse strand (blue).
Using the INSDC location string notation, these regions are
``\texttt{1050..2080}'' and ``\texttt{complement(1050..2080)}''
respectively if implicitly given in terms of the reference chr1.
Using the GTF/GFF3 family of formats, regardless of the
strand these two locations are described with $start = 1050$
and $end = 2080$, and in general, $start \leq end$.
Biologically speaking, the start is strand dependent.
For the forward strand feature (orange), the start is 1050
while the reverse strand feature (blue) starts from 2080.
\textit{TODO - Replace with real example? Add FALDO illustration?}
}
\label{fig:strands}
\end{figure}

\section*{Results}
One of the practical goal driving the development of FALDO was to be able
to represent all the annotated sequences in UniProt and the INSDC as RDF
triples, as a step towards providing this data via SPARQL end points whereby
it can be queried.
UniProt is already available as RDF, and a revised version of the full
dataset using FALDO has already been tested and is expected to be included
in the UniProt 2013\_11 release \textit{TODO - fill in version and data}.

The protein examples considered, such as the UniProt feature annotations,
require relatively simple locations within protein sequences to be described (see Fig:\ref{fig:UniProt}).
There are corner cases for situations like unknown boundaries for these we use the $faldo:InRangePosition$.

\begin{figure}
\begin{verbatim}
AC   P05064;
...
FT   ACT_SITE    230    230       Schiff-base intermediate with
FT                                dihydroxyacetone-P.
\end{verbatim}
\begin{verbatim}
uniprot:P05064 a up:Protein ;
   up:annotation SHA:E50..1B .
SHA:E50..1B a up:Active_Site_Annotation ;
   rdfs:comment ``Schiff-base intermediate with dihydroxyacetone-P''
   faldo:location range:22571003137111085tt230tt230 .
range:22571003137111085tt230tt230 a faldo:Position ;
   faldo:position 230 ;
   faldo:reference isoform:P05064-1 .
\end{verbatim}
\caption{Example from UniProt entry P05064 showing the position of an active site in both faldo and the UniProt flat-file format.}
\label{fig:UniProt}
\end{figure}
\subsection*{Complement strand}
Feature locations on nucleotide sequences can be relatively simple,
%GenBank
for the example (shown in Fig:\ref{fig:insdcComplement}) the \textit{cheY} gene in
\textit{Escherichia coli} str. K-12 substr. MG1655 (accession NC\_000913.2)
is described in the INSDC feature table as \texttt{complement(1965072..1965461)},
which is 390 base pairs using inclusive one-based counting.
This feature begins on the base complementary to $start = 1965461$
and finishes at $end = 1965072$, and so the INSDC location string
can be interpreted as \texttt{complement($end$..$start$)},
FALDO respects this biological interpretation of a feature location
on the reverse strand:

%Notice begin, end = 1965461, 1965072 (biological interpretation)
\begin{figure}
\begin{shaded}
\small
\begin{verbatim}
# 
@prefix faldo: <http://biohackathon.org/resource/faldo#> .

# Here are example triples describing a gene, note the location line:
<_:geneCheY> a so:Gene ;
           rdfs:label "cheY" ;
           faldo:location <_:example> ;

# The following triples define the location used by the feature above,
# made up of a region and its two positions:
<_:example> a faldo:Region ;
           faldo:begin <_:example_b> ;
           faldo:end <_:example_e> .

<_:example_b> a faldo:Position ; 
           a faldo:ExactlyKnownPosition ;
           a faldo:ReverseStrandPosition ;
           faldo:position "1965461"^^xsd:int ;
           faldo:reference refseq:NC_000913.2 .

<_:example_e> a faldo:Position ; 
           a faldo:ExactlyKnownPosition ;
           a faldo:ReverseStrandPosition ;
           faldo:position "1965072"^^xsd:int ;
           faldo:reference refseq:NC_000913.2 .
\end{verbatim}
\end{shaded}
\caption{Incomplete example of using Faldo in turtle format to describe
gene chrY being at complement(NC\_$000913.2:1965072..1965461$)}
\label{fig:insdcComplement}
\end{figure}

In contrast, other formats like GFF require $start \leq end$
regardless of the strand, which is equivalent to interpreting
the INSDC location string as \texttt{complement($start$..$end$)}.
This convention has a number of practical advantages when
dealing with numerical operations on features sets, such as
checking for overlaps or indexing data. For example, the
feature length is given by $length = end - start + 1$ under
this numerically convenient scheme where the interpretation
of $start$ versus $end$ is strand independent.



\subsection*{Compound locations}
Nucleotide data from the INSDC (or GFF3) can require considerably
more complicated sequence locations built out of multiple-regions.

\textit{TODO - examples}.

\subsection*{Fuzzy locations}
Nucleotide data from the INSDC can also require ambiguous end points.

\textit{TODO - examples}.

All of the UniProt protein sequences are now available in RDF format,
with a SPARQL end point currently in beta. Converting more recent
curated INSDC data has been straightforward, however there are a
number of deprecated location strings which have been problematic.
\textit{TODO - examples? manual curation to fix last few oddities?}

\subsection*{Restriction enzymes}

The recognition sites of most restriction enzymes are relatively
straightforward to describe, as is the cleavage site of a blunt
end cutting enzyme \textit{TODO - example of this? Between location}.

However, the cutting point of sticky-end cutting enzymes which
leave an overhang are more challenging to describe. Here the
flexibility in FALDO to allow the start and end positions of a region
to have different strands can be used (Figure~\ref{fig:HindIII}).

\begin{figure}
\begin{framed}
\small
\begin{verbatim}
    \/
5'..AAGCTT..3' 
    123456
3'..TTCGAA..5'
        /\
        
:EnzymeRestrictionSite a example:EnzymeRestrictionSite ;
       faldo:location :EnzymeRestrictionRegion .
       
:EnzymeRestrictionRegion a faldo:Region ;
       faldo:begin :53_1 ;
       faldo:end   :35_6 .
       
:53cleaveageSite a example:CleavageSite ;
       faldo:location :53cleaveageSitePosition .
:35cleaveageSite a example:CleavageSite ;
       faldo:location :35cleaveageSitePosition .
        
:53cleaveageSitePosition a faldo:InBetweenPosition, SO:0001689  ; #five_prime_restriction_enzyme_junction 
       faldo:after :53_1 ;        
       faldo:before :53_2 .
       
:35cleaveageSitePosition a faldo:InBetweenPosition, SO:0001690 ; #three_prime_restriction_enzyme_junction 
       faldo:before :35_5 ;
       faldo:after :35_6 . #Note the before after reversion due to biological start      

       
:53_1 a faldo:ExactPosition , faldo:ForwardStrandedPosition ;
      faldo:position 1 .

:53_2 a faldo:ExactPosition , faldo:ForwardStrandedPosition ;
      faldo:position 2 .

:35_5 a faldo:ExactPosition , faldo:ReverseStrandedPosition ;
      faldo:position 5 .

:35_6 a faldo:ExactPosition , faldo:ReverseStrandedPosition ;
      faldo:position 6 .
\end{verbatim}
\end{framed}
\caption{HindIII restriction enzyme cleavage site with sticky ends}
\label{fig:HindIII}
\end{figure}

Circular genomes bring have number of interesting features to describe. 
The most common is  a feature that straddles the origin of replication or the '0' mark.
For example a gene encoded on the reverse strand of the f1 bacteriophage \textit{todo find accession} which is 6407 base pairs long.
GeneII transcription starts from coordinate 6006 and the transcription ends at position 831.


\subsection*{More examples}

\textit{TODO - other ideas:}
\begin{itemize}
\item Example of Cys bonding between one or two proteins?
\item RNA/DNA digest sites, and the cut sites -- especially for over hangs (start and ends on opposite strands)
\item tRNA hairpin (two complementary regions)?
\item $\beta$-sheet in protein?
\item RNA-Seq mapping (useful for SAM/BAM)?
\end{itemize}



\section*{Discussion}
When designing FALDO, a broad range of use cases were considered from
human genome annotation to glycan binding sites and protein domains on
amino acid sequences, with the goal of developing a scheme general enough
to describe regions of DNA, RNA and protein sequences.

Advantages and drawbacks of existing file formats were considered, including
line based column formats like BED and GTF/GFF which focus on exact
ranges on a given sequence, and the more complex locations supported
by the INSDC feature tables used by GenBank/EMBL/DDBJ.

The simplest non-stranded range location on a linear sequence requires
a start and end coordinate, but even here there are existing competing
conventions for describing open or closed end points using zero and
one based counting (for example BED versus GTF/GFF3/INSDC).

Similarly multiple schemes exist for describing strand specific locations,
with some formats describing features on the reverse strand implicitly
when the start coordinate is numerically higher than the end coordinate
\textit{(TODO - example of format which does this)}.
In FALDO we always count from the start of the forward 5'-3' strand even on the reverse strand.
This encoding means there is no need to know the length of the sequence to compare positions on the different strands.

For a semantic description describing the strand explicitly is preferable.
FALDO chooses to add the strand information to the position. 
This is required to accurately describe for example the sticky ends
of an enzyme digestion cut site, as in the HindIII example (Figure~\ref{fig:HindIII}).

Unlike formats like GTF/GFF3, FALDO adopts the convention
that the start coordinate should be the biological start which 
may be numerically a higher value than the end coordinate.

A major difference with other proposed standards is that we chose to make strandedness and reference sequence a property of the position instead of the region.
This is important in a number of use cases.
One is when we someone needs to describe the position of a Gene on a rough assembly where the start and end are known to be on a different contig. 
This can be the case when RNA mapping is used in the genome assembly process.
Another is when rough semantics are used in queries e.g. answering what is the start and end of a Gene. 
In a process called transplicing exons of one gene can be found on multiple chromosomes e.g. \textit(TODO find an example)
In such a case the start of the gene is on a different reference sequence than the end.
We encourage that data providers describe the biology more precise than gene start-end in their databases, but we know that legacy data is not easily converted.
Another is Disulfide bonds in a protein complex the start position is on one chain while the end is on another chain in the complex.
FALDO can describe the position of the bridge with InBetweenPositions where the start and end of the bridge is on different protein chains.
This would not be possible if the reference sequence was a property of the region. 
The last difference is that it makes queries more predictable. 
Current formats describe single amino acid binding sites as regions.
FALDO allows us to describe such sites directly on positions therefore the Position class needs the reference sequence property.
For the Region class it is not required as a Region always ends and starts with Positions.

\subsection*{OWL2 reasoning based compression}
For large databases such as INDSC or UniProt the need to repeat the reference sequence for positions again and again incurs a significant cost in storage.
OWL2 reasoning introduced a new option called PropertyChain reasoning. 
With the following axioms the faldo:reference triples can be inferred for any position described by a INSDC record.
\begin{verbatim}
INDSC:reference
      a       owl:ObjectProperty ;
      rdfs:subPropertyOf faldo:reference ;
      owl:propertyChainAxiom
              (faldo:beginOf faldo:endOf :featureOf insdc:sequence) .
\end{verbatim}
A OWL capable query rewriter (e.g. such as part of Stardog) means that users do not see the difference between encoding the faldo:reference explicitly or having them inferred at query time.
Other stores that do not have such capabilities can easily add the necessary triples using a single SPARQL insert query.
\begin{verbatim}
CONSTRUCT {
    ?position faldo:reference ?sequence .
}
WHERE {
    ?record a insdc:Entry .
    ?record insdc:feature ?feature .
    ?feature faldo:position ?position .
    ?record insdc:sequence ?sequence .
}
\end{verbatim} 
For a SPARQL mapping to legacy files such as SPARQL-bed these extra triples do not matter as they do not exist in any way on disc .


FALDO has been designed to be open to extension for future
use \textit{TODO - expand on this with Jerven}.

\section*{Conclusions}
FALDO is a simple and minimalistic ontology for describing biological features in a consistent manner that Bio-informaticians can depend upon.
The diverse software and databases using FALDO show that it has enough power to describe all biological feature positions.
It is now much easier for users querying biological databases on the semantic web to compare features on the basis of locations.
This also means that visualisation tools that access position data via SPARQL can easily reuse significant parts of queries between databases.
\section*{Availability and requirements}
FALDO is publicly available at the URL \url{http://biohackathon.org/resource/faldo}
which is developed under source code control at
\url{https://github.com/JervenBolleman/FALDO} hosted by GitHub Inc,
where everyone is free to suggest extensions and improvements and if required extend FALDO to meet their unique requirements.
FALDO currently uses the Creative Commons Attribution Zero 1.0 Public Domain dedication license,
making FALDO available to use and reuse free of charge.

The ontology is shared in the Turtle (\url{http://www.w3.org/TR/turtle/}) RDF syntax,
which can be automatically converted to another RDF syntax such as RDF/XML if required.


\section*{List of abbreviations}
\begin{description}
\item[SPARQL] SPARQL Protocol and RDF Query Language
\item[UniProtKB] Universal Protein Knowledgebase 
\item[RDF] Resource Description Framework
\item[DNA] Deoxyribonucleic acid
\end{description}
\bigskip

\section*{Authors contributions}

Jerven Bolleman wrote the basic ontology file and mapping to UniProt RDF.
Robert Hoehndorf did the GFF3 to OWL conversion.
Peter Cock adapted the Biopython data object model to this format (TODO).
Raoul JP Bonnal adapted BioRuby to match the ontology and wrote the cufflink to locations.rdf converter with Francesco Strozzi.

Totokama Fujisawa implemented the DDBJ RDF in terms of this format.
Robert Beuels adapted JBrowse to query SPARQL endpoints that use this format to generate custom tracks. 

\section*{Acknowledgments}



\newpage
{\ifthenelse{\boolean{publ}}{\footnotesize}{\small}
 \bibliographystyle{bmc_article}  % Style BST file
  \bibliography{locations} }     % Bibliography file (usually '*.bib' ) 
  
\end{bmcformat}
\end{document}

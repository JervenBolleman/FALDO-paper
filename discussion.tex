\section*{Discussion}
When designing FALDO, a broad range of use cases were considered from
human genome annotation to glycan binding sites and protein domains on
amino acid sequences, with the goal of developing a scheme general enough
to describe regions of DNA, RNA and protein sequences.

Advantages and drawbacks of existing file formats were considered, including
line based column formats like BED and GTF/GFF which focus on exact
ranges on a given sequence, and the more complex locations supported
by the INSDC feature tables used by GenBank/EMBL/DDBJ.

The simplest non-stranded range location on a linear sequence requires
a start and end coordinate, but even here there are existing competing
conventions for describing open or closed end points using zero and
one based counting (for example BED versus GTF/GFF3/INSDC).

Similarly multiple schemes exist for describing strand specific locations,
with some formats describing features on the reverse strand implicitly
when the start coordinate is numerically higher than the end coordinate
\textit{(TODO - example of format which does this)}.
In FALDO we always count from the start of the forward 5'-3' strand even on the reverse strand.
This encoding means there is no need to know the length of the sequence to compare positions on the different strands.

For a semantic description describing the strand explicitly is preferable.
FALDO chooses to add the strand information to the position. 
This is required to accurately describe for example the sticky ends
of an enzyme digestion cut site, as in the HindIII example (Figure~\ref{fig:HindIII}).

Unlike formats like GTF/GFF3, FALDO shares with Chado the convention
that the start coordinate should be the biological start (which 
may be numerically a higher value than the end coordinate).

A major difference with other proposed standards is that we chose to make strandedness and reference sequence a property of the position instead of the region.
This is important in a number of use cases.
One is when we someone needs to describe the position of a Gene on a rough assembly where the start and end are known to be on a different contig. 
This can be the case when RNA mapping is used in the genome assembly process.
Another is when rough semantics are used in queries e.g. answering what is the start and end of a Gene. 
In a process called transplicing, exons of one gene can be found on multiple chromosomes, or on different strands of the same chromosome. e.g. \textit(TODO find an example)
In such a case the start of the gene is on a different reference sequence or strand than the end.
We encourage that data providers describe the biology more precise than gene start-end in their databases, but we know that legacy data is not easily converted.
Another is Disulfide bonds in a protein complex the start position is on one chain while the end is on another chain in the complex.
FALDO can describe the position of the bridge with InBetweenPositions where the start and end of the bridge is on different protein chains.
This would not be possible if the reference sequence was a property of the region. 
The last difference is that it makes queries more predictable. 
Current formats describe single amino acid binding sites as regions.
FALDO allows us to describe such sites directly on positions therefore the Position class needs the reference sequence property.
For the Region class it is not required as a Region always ends and starts with Positions.

% this could also be integrated with the above rather than forming its own section
\subsection*{Comparison with GMOD/Chado}

The GMOD (Generic Model Organism Database) schema
Chado\cite{Chado2007} includes a module for representing sequences and
sequence features. The Chado model has similarities with RDF in that
the subject-predicate-object pattern is used to represent connections
between biological entities, with community ontologies such as SO used
to type entities and predicates.

Like FALDO, Chado uses the convention that the start position is
always the biological start, even on the reverse strand. Chado allows
multiple locations per feature, but unlike FALDO, the start and end of
any location must be in the same region, which prohibits for example a
feature that spans more than one contig.

The Chado triple pattern is layered on top of a conventional
relational database which leads to inefficiencies when performing many
queries due to the large number of joins required. An RDF store
provides a more natural fit for this kind of model.

\subsection*{Efficiency of Region-of-Interest queries}

Region of interest (ROI) queries are common operations performed on a
set of genome annotations to extract a set of features within a
range. For applications such as genome browsers, it is important that
these are as efficient as possible. Many RDF query engines will
perform poorly when performing ROI queries over large feature
sets. However, we expect that it should be possible to combine
efficient algorithms and indexes such as Nested Containment Lists
(NCLs)\cite{NCL2007} or spatial indexes to optimize these operations
in the context of a SPARQL query.



\subsection*{OWL2 reasoning based compression}
For large databases such as INDSC or UniProt the need to repeat the reference sequence for positions again and again incurs a significant cost in storage.
OWL2 reasoning introduced a new option called PropertyChain reasoning. 
With the axiom shown in fig:\ref{owl:chainProperty} the faldo:reference triples can be inferred for any faldo:position described by a INSDC record.
\begin{figure}
\begin{verbatim}
INDSC:reference
      a       owl:ObjectProperty ;
      rdfs:subPropertyOf faldo:reference ;
      owl:propertyChainAxiom
              (faldo:endOf faldo:locationOf INDSC:featureOf INDSC:sequence) , 
              (faldo:locationOf INDSC:featureOf INDSC:sequence) , 
              (faldo:beginOf faldo:locationOf INDSC:featureOf INDSC:sequence) .

\end{verbatim}
\caption{OWL2 property chain axiom to infer that all positions described in a INSDC record are positioned relative to the main sequence of the record.}
\label{owl:chainProprty}
\end{figure}
A OWL capable query rewriter (e.g. such as part of Stardog) means that users do not see the difference between encoding the faldo:reference explicitly or having them inferred at query time.
Other stores that do not have such capabilities can easily add the necessary triples using a single SPARQL insert query as shown in fig:\ref{sparql:chainProperty}.
\begin{figure}
\begin{verbatim}
CONSTRUCT {
    ?position faldo:reference ?sequence .
}
WHERE {
    ?record a insdc:Entry .
    ?record insdc:feature ?feature .
    ?feature faldo:location ?location .
    {
        ?location faldo:begin|faldo:end ?position .
    }
    UNION
    {
        ?location a faldo:Position .
    } .
    ?record insdc:sequence ?sequence .
}
\end{verbatim}
\caption{SPARQL query to add all faldo:reference properties to faldo:positions described from a insdc:record.}
\label{sparql:chainProperty}
\end{figure}

FALDO has been designed to be open to extension for future use.
To avoid unnesecarry contradictions we avoid stating that a position is unknown.
i.e. a missing triple means unknown. 
This means that a new datasource with more specific information is not contradicted e.g.
we never get $\_:position a :faldo:ExactPosition,faldo:UnknownPosition$ when merging rdf sources.



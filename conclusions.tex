\section*{Conclusions}
FALDO is a simple and minimalistic ontology for describing biological features in a consistent manner that bioinformaticians can depend upon.
The diverse software and high-profile databases already using FALDO show that it has enough power to describe all biological feature positions.
The uptake of this ontology means that it is now much easier for users querying biological databases on the semantic web to compare features on the basis of locations.
This also means that visualisation tools that access position data via SPARQL can easily reuse significant parts of queries between databases.

We know that legacy data is not easily converted, but we encourage data providers to describe the known biology as precisely as possible in their databases.

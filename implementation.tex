\section*{Implementations}
FALDO was implemented and used in a number of tools and databases.

\begin{description}
\item[JBrowse sparql] FALDO is used sparql queries determine where to draw information related to a reference sequence in tracks filled with data from semantic databases. 
\item[UniProt] UniProt annotates many protein features and sites. From release 2013\_10 in the RDF rendering of UniProt this is done with FALDO.
\item[INSDC-DDBJ] DDBJ is currently working on a RDF format for the INSDC data that is stored in DDBJ/GenBank/EMBL-Bank.
\item[Biointerchange] uses FALDO to make position information in current bioinformatics data stored in files such as GFF3 and GVF available to the semantic web.
\item[TogoGenomes] a bacterial genome database collection provided by the DBCLS also uses FALDO in its RDF representation.
\item[Intermine] The popular model organism database software collection uses FALDO in its SPARQL mode.
\item[phenomebrowser] Positions of phenotypes and disease related natural variations are positioned onto the Mouse genome using FALDO.
\item{ENSEMBL-R2RML] The open source conversion layer to make the ENSEMBL mysql databases available on the semantic web also uses FALDO to describe all feature locations.
%\cite(e) 
\item[SPARQL-bed] This simple tool that turns any BED file into a web accessible SPARQL endpoint also uses FALDO to describe BED feature positions.
\end{description}

The core ontology is described in the RDF rendering of OWL with minimal extensions for quality control in INSDC and UNIPROT databases.
It is maintained as open source using the Creative-Commons 3.0 Attribution license in a git repository where anyone can raise issues or bug reports 
as well as request changes using pull requests.
The ontology is available via purl hosted at the biohackathon.org website.






